\chapter{Questions}
\label{chap:questions}

To help the student to find a red line to focus its research on, he was required work on the subject "What should be the initial questions to asks to start making \gls{agi} Chatbots" as preliminary study before the beginning of deepening project itself and to write down the outcome as a set of questions related to his interests and the field of \gls{agi} Chatbots.

\section{Initial and Broad Questions}
As a result to the preliminary study, the following question were produced. Please take into account that those questions were not meant to be answered as part of the project itself, but as part of the process of appropriation of the field of study.

\begin{itemize}[noitemsep]
    \item Is the \gls{ann} approach appropriate to represent the world?
    \item Can agents be made exclusively from a language?
    \item Can agents able to experience an environment?
    \item Is a narrative environment be enough to understand an environment?
    \item Is the language able to provide to an agent an understanding of the world?
    \item Is the knowledge of the language syntax enough to gain an understanding?
    \item Is the result of unsupervised learning enough to discover all nuances?
    \item Is the unsupervised learning sufficient to make sense to an environment?
    \item Is a descriptive explanation of the world be expressed in a language?
    \item Is the description good enough to catch all the nuances?
    \item Is the language good enough to explain?
    \item Can we augment or make a semantic language?
    \item Can we create a common symbolic language?
    \item Is the language multi-dimensional?
    \item How many dimensions are needed for a complex language?
    \item Is it possible to give a word equivalence to machines for human-specific words?
    \item Are all emotions describable into words?
    \item Are emotions altering language descriptions?
    \item Is an approximation of the real world enough to understand the environment?
    \item Would a the simulated world be a good approximation of the real world?
\end{itemize}

\section{Narrowed Questions}
In a second time, the student was asked to narrow the initial questions above into potential fields of study.

\begin{itemize}
    \item Common human-machine language
    \begin{itemize}[noitemsep]
        \item Is it possible to create a multi-dimensional human-machine language, which includes a common semantic, symbolic, and emotion definition.
        \item Is it possible to create an abstract world for machines to understand human symbolic based on a real world, and define fundamentals for machine representation of the language.
    \end{itemize}
    \item Machine intuition
    \begin{itemize}[noitemsep]
        \item Is it possible to provide to machines an human-like intuition (inside voice), which would help to keep a long term context and specialize in specific fields.
    \end{itemize}
    \item Evaluate human-machine communication
    \begin{itemize}[noitemsep]
        \item Is it possible to provide a protocol to test the communication skills and machine understanding.
    \end{itemize}
\end{itemize}


\section{Potential Red lines}
From the potential fields above, the following suggested red lines were proposed.

\begin{itemize}[noitemsep]
    \item How to verify and quantify a chatbot understanding?
    \item What is the premise to make chatbots general with today's technology?
    \item How chatbot can be proactive?
    \item How to simulate human-like intuition in chatbots?
\end{itemize}


\section{The Deepening Project Question and Red line}
Based on reflective work and discussions, the concluding red line and question for this deepening project is:

\begin{itemize}[noitemsep]
    \item What is Word Embedding and how is it useful for chatbots?
\end{itemize}


