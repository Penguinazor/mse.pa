% ---------------------------------------------------------------------
% Cloned from the HES-SO//Master canvas 2019
% ---------------------------------------------------------------------
\chapter{Introduction}
\label{chap:introduction}

Beginning of 2019, chatbots are everywhere but very limited to narrow tasks, and are, in most cases, sequences of if-else conditions resulting in a very weak AI. Indeed, hard-coded connections are requiring an infinite amount of human power to create generic Chatbots able to maintain a conversation at a human level. However, the progress in the field of machine learning is demonstrating that providing large corpora to an unsupervised algorithm is enough to maintain a passive conversation with users, which results into a shifting of the human power into data engineering. Multiple algorithms and technics are emerging monthly, which are demonstrating promising conversational performance improvement; however, they are all still narrow AIs. Indeed, even if they are getting better at providing meaningful sentences, they are still not able to generalize all tasks linked to a conversation, such as, understanding the context, search and learn for missing information, initiate conversation in a meaningful manner, be intuitive, and more. The generalization of those features would allow a significant step forward into general Chatbots.

As the driver, iCoSys, the Institut of Complex Systems at University of Applied Sciences and Arts at Fribourg, Switzerland, is interested into the result of this project as a study for their AI-News project, whose goal is to provide a chatbot as a tool to reader, to help them narrow their interests and deliver the right information. AI-News is in collaboration with the Swiss Innovation Agency from the Swiss Confederation, and La Liberté, the daily newspaper from Fribourg. 

\section{Aim of Study}
In harmony with the author interest, the goal of this deepening semester project is to suggest and demonstrate strategic approaches as a premise to the Artificial General Intelligence (AGI) and getting a step closer to general Chatbots, which can initiate and maintain human-like conversations in a pro-active manner.

\section{Scope and Study Borders}
As a red line for this deepening project, the focus will be on the Word2Vec technology, from a research perspective. Indeed, this technology is seen as a foundation for the modern Natural Language Processing (NLP) and Deep Neural Network (DNN) Chatbots, which makes it an exciting vector of study about its current usage, its extensions, and potential evolution. 