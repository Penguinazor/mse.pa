\chapter{Plan}
%\label{chap:plan}


\section{Contraints}
\textbf{Timeframe:} 15 weeks\\
\textbf{Starting date:} 18.02.2019\\
\textbf{Ending date:} 31.05.2019

\section{Initial Plan}
As the first milestone for the deepening project, the student was required to create an initial plan, with the purpose to help himself and the teacher to visualize the project's main red line.
\subsection{Tasks}
\begin{enumerate}
    \setlength\itemsep{0em}
    \item Initial research about general chatbots
    \item Determine the project target
    \item Play with the subject
    \item Explore the Word2Vec methodology
    \item Explore the Word2Vec extensions
    \item Combine and test \acrshort{ann} algorithms with Word2Vec
    \item Explore ANN algorithm topology for the chatbot
    \item Analyze of the chatbot intuition with parallel algorithms
    \item Analyze of a protocol to evaluate proactive chatbots
    \item Analyze Profile-based initiatives
    \item Analyze and experiment profile nurturing 
    \item Analyze and experiment with chatbot initiatives with no profiles
    \item Make overall improvements
    \item Autonomous data gathering
    \item Make suggestions
    \item Determine possible continuation and future outcomes for the project
\end{enumerate}

\subsection{Milestones}
\begin{enumerate}
    \setlength\itemsep{0em}
    \item Initial deepening project plan and specification document
    \item Basic multi-dimensional word embedding space
    \item Basic conversational agent
    \item Basic proactive chatbot
    \item Deepening project report
\end{enumerate}

\subsection{Sprints}

\paragraph{18.02.19 to 08.03.19} (3 weeks) 
\begin{itemize}
    \setlength\itemsep{0em}
    \item Do the initial research about general chatbots
    \item Determine the project target
    \item Play with the subject
    \item \textbf{DELIVERABLE:} Plan and Initial Specification document
\end{itemize}

\paragraph{11.02.19 to 29.03.19} (3 weeks)
\begin{itemize}
    \setlength\itemsep{0em}
    \item Explore the Word2Vec methodology and its extensions
    \item Combine and test \acrshort{ann} algorithms with Word2Vec
    \item \textbf{MVP:} Basic multi-dimensional word embedding space
\end{itemize}

\paragraph{01.04.19 to 19.04.19} (3 weeks)
\begin{itemize}
    \setlength\itemsep{0em}
    \item Explore \acrshort{ann} algorithm topology for the chatbot
    \item Analysis of the chatbot intuition with parallel algorithms
    \item Analysis of a protocol to evaluate proactive chatbots
    \item \textbf{MVP:} Basic conversational agent
\end{itemize}

\paragraph{22.04.19 to 10.05.19} (3 weeks)
\begin{itemize}
    \setlength\itemsep{0em}
    \item Profile-based initiatives
    \item Analysis and experiment of the profile nurturing 
    \item Analyze and experiment with chatbot initiatives with no profiles.
    \item \textbf{MVP:} Basic proactive chatbot
\end{itemize}

\paragraph{13.05.19 to 31.05.19} (3 weeks)
\begin{itemize}
    \setlength\itemsep{0em}
    \item Overall improvements
    \item Autonomous data gathering
    \item Make suggestions
    \item Determine possible continuation and future outcomes for the project
    \item \textbf{DELIVERABLE:} Report + Sources
\end{itemize}

\subsection{Gantt chart}
Figure ~\ref{fig:gantt-initial} represents the visual gantt chart for the initial plan.

\newganttchartelement*{mymilestone}{
mymilestone/.style={
shape=isosceles triangle,
inner sep=0pt,
draw=cyan,
top color=white,
bottom color=cyan!50
},
mymilestone incomplete/.style={
/pgfgantt/mymilestone,
draw=yellow,
bottom color=yellow!50
},
mymilestone label font=\slshape,
mymilestone left shift=0pt,
mymilestone right shift=0pt
}

\newgantttimeslotformat{stardate}{
\def\decomposestardate##1.##2\relax{
\def\stardateyear{##1}\def\stardateday{##2}
}
\decomposestardate#1\relax
\pgfcalendardatetojulian{\stardateyear-01-01}{#2}
\advance#2 by-1\relax
\advance#2 by\stardateday\relax
}

\begin{figure}[h]%[htbp]
\centering
%\begin{sidewaysfigure}
\begin{ganttchart}[vgrid, hgrid]{1}{16}
\gantttitle{February}{4} 
\gantttitle{March}{4}
\gantttitle{April}{4}
\gantttitle{May}{4}\\
\gantttitlelist{1,...,16}{1}\\

%Sprint 1
\ganttgroup{Sprint 1}{3}{5} \\
\ganttbar{Tasks 1, 2, 3}{3}{5} \\
%\ganttlink{elem1}{elem2}
\ganttmilestone{Milestone 1}{5}\\

%Sprint 2
\ganttgroup{Sprint 2}{6}{8} \\
\ganttbar{Task 4, 5}{6}{7} \\
\ganttbar{Task 6}{8}{8} \\
\ganttmilestone{Milestone 2}{8}\\
\ganttlink{elem2}{elem3}
\ganttlink{elem4}{elem5}

%Sprint 3
\ganttgroup{Sprint 3}{9}{11} \\
\ganttbar{Tasks 7, 8, 9}{9}{11} \\
\ganttmilestone{Milestone 3}{11}\\
\ganttlink{elem6}{elem7}

%Sprint 4
\ganttgroup{Sprint 4}{11}{13} \\
\ganttbar{Tasks 10, 11, 12}{11}{13} \\
\ganttmilestone{Milestone 4}{13}\\
\ganttlink{elem9}{elem10}

%Sprint 5
\ganttgroup{Sprint 5}{14}{16} \\
\ganttbar{Tasks 13, 14, 15, 16}{14}{16} \\
\ganttmilestone{Milestone 5}{16}\\
\end{ganttchart}
%\end{sidewaysfigure}

\caption{Initial Gantt Chart}
\label{fig:gantt-initial}
\end{figure}


\section{Effective Plan}
As expected the initial plan served as an initial model, and evolved iteratively based on the student and teacher feedback while exploring the subject.
\subsection{Tasks}
\begin{enumerate}
    \setlength\itemsep{0em}
    \item Initial research about general chatbots
    \item Determine the project target
    \item Set the initial plan
    \item Make LaTeX report template
    \item Explore the Word2Vec subject
    \item Explore the Word2Vec algorithm
    \item Build a Word2Vec model on the latest english wikipedia dump
    \item Explore Word2Vec parameters
    \item Explore Word2Vec analogies
    \item Explore Word2Vec sentence generation
    \item Explore visual representations of Word2Vec vectors
    \item Explore Word2Vec applications with chatbots
    \item Write the report
    
\end{enumerate}

\subsection{Milestones}
\begin{enumerate}
    \setlength\itemsep{0em}
    \item Initial deepening project plan and specification document
    \item Basic Word2Vec Word Embedding Model
    \item Conclusions Word2Vec based chatbots
    \item Ideas to make chatbots proactive
    \item Deliver the report
\end{enumerate}

\subsection{Gantt chart}
Figure ~\ref{fig:gantt-effective} represents the visual gantt chart for the effective plan.

\begin{figure}[h]%[htbp]
\centering
%\begin{sidewaysfigure}
\begin{ganttchart}[vgrid, hgrid]{1}{16}
\gantttitle{February}{4} 
\gantttitle{March}{4}
\gantttitle{April}{4}
\gantttitle{May}{4}\\
\gantttitlelist{1,...,16}{1}\\

\end{ganttchart}
%\end{sidewaysfigure}

\caption{Effective Gantt Chart}
\label{fig:gantt-effective}
\end{figure}
